\question{(B5.2)}{
    Prove that the matrix
    \[
        A =
        \begin{pmatrix}
            \transpos{X} X + K I_n & \transpos{X}\mathbf{1}_m\\
            \transpos{\mathbf{1}_m} X& m
        \end{pmatrix}
    \]
    is symmetric positive definite.
    You can either use an argument involving a Schur complement
    (Chapter 7 of Vol II, linalg-II),
    or proceed as follows.\\
    \medskip
    Let $g$ be the function given by
    \begin{align*}
        g(w,b)  & =
        \begin{pmatrix}
            \transpos{w} & b
        \end{pmatrix}
        \begin{pmatrix}
            \transpos{X} X + K I_n  & \transpos{X}\mathbf{1}_m \\
            \transpos{\mathbf{1}_m} X & m
        \end{pmatrix}
        \begin{pmatrix}
            w \\
            b
        \end{pmatrix} \\
        & =
        \transpos{w}(\transpos{X} X + K I_n) w +
        2 \transpos{w} \transpos{X}\mathbf{1}_m b
        +   \transpos{\mathbf{1}_m}\mathbf{1}_m b^2.
    \end{align*}
    Then
    $A$ is symmetric positive definite iff
    $(w, b) \not= 0$  implies that
    $g(w, b) > 0$.
    Prove that if $w\not= 0$ and $b = 0$,  then  $g(w, 0) > 0$.
    If $b\not= 0$, for $b$ fixed
    the function $w \mapsto g(w, b)$ is strictly convex
    because $\transpos{X} X + K I_n$ is SPD, so it has a unique minimum
    obtained by setting the gradient $\nabla_w g$ to $0$.
    Find the value $w^*$ for which $\nabla_w g = 0$, and
    compute the corresponding minimum value $g(w^*, b)$.
    Prove that $g(w^*,b)$  is of the form $Tb^2$ and compute $T$
    ($T$ happens to be
    the Schur complement of $\transpos{X} X + K I_n$ in $A$).
    Prove that $T> 0$, so that if $b\not= 0$, then $g(w^*, b) > 0$.
    Deduce that $A$ is symmetric positive definite.
}
\begin{proof}
    Since A is symmetric, we can take A to be
    \[
        A =
        \begin{pmatrix}
            \transpos{X} X + K I_n & \transpos{X}\mathbf{1}_m\\
            \transpos{\mathbf{1}_m} X& m
        \end{pmatrix}
        =
        \begin{pmatrix}
            T & U \\
            \transpos{U} & V
        \end{pmatrix}
    \]
    and use the Schur complement to check if it is SPD. Let $B$ be the Schur complement of
    $T$. We can then see
    \begin{align*}
        B & = V - \transpos{U}A^{-1}U
        \\ & = m - \transpos{\mathbf{1}}_mX(\transpos{X}X + KI_n)^{-1}\transpos{X}\mathbf{1}_m
        \\ & = \transpos{\mathbf{1}}_m\mathbf{1}_m - \transpos{\mathbf{1}}_mX(\transpos{X}X + KI_n)^{-1}\transpos{X}\mathbf{1}_m
        \\ & = \transpos{\mathbf{1}}_m(I_m - X(\transpos{X}X + KI_n)^{-1}\transpos{X})\mathbf{1}_m
    \end{align*}
    and since we showed in B4 that
    \[
        (\transpos{X} X + K I_n)^{-1} \transpos{X}  =  \transpos{X} (X \transpos{X} + K I_m)^{-1}
    \]
    we can say
    \begin{align*}
        B & = \transpos{\mathbf{1}}_m(I_m - X(\transpos{X}X + KI_n)^{-1}\transpos{X})\mathbf{1}_m
        \\ & = \transpos{\mathbf{1}}_m(I_m - X\transpos{X}(\transpos{X}X + KI_n)^{-1})\mathbf{1}_m
        \\ & = \transpos{\mathbf{1}}_m(\transpos{X}X(\transpos{X}X + KI_n)^{-1} + KI_n(\transpos{X}X + KI_n)^{-1} - X\transpos{X}(\transpos{X}X + KI_n)^{-1})\mathbf{1}_m
        \\ & = \transpos{\mathbf{1}}_m(KI_n(\transpos{X}X + KI_n)^{-1})\mathbf{1}_m
        \\ & = \transpos{\mathbf{1}}_m(K(\transpos{X}X + KI_n)^{-1})\mathbf{1}_m.
    \end{align*}
    Since, we were given that $\transpos{X}X + KI_n$ is SPD in B4, then $B$ must also be SPD, since
    $B = \transpos{\mathbf{1}}_m(K(\transpos{X}X + KI_n)^{-1})\mathbf{1}_m$. Finally, because $\transpos{X}X + KI_n$ and
    $B$ are SPD, then by \prop{7.4} (Vol. II), $A$ is SPD as well.
\end{proof}
\newpage
