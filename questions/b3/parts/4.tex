\question{(B3.4)}{
    Proposition 3.5 (Vol II) shows that for any continuous
    bilinear map $\mapdef{f}{E_1\times E_2}{F}$,
    for every $(a, b) \in E_1\times E_2$, the derivative
    $\mathrm{D} f_{(a, b)}$ exists and is given by
    \[
        \mathrm{D} f_{(a, b)} (u, v) = f(u, b) + f(a, v),
    \]
    for all $(u, v) \in E_1\times E_2$.\\
    \medskip
    It can be shown (and you need not prove it, unless you decide to solve
    the extra credit problem) that for any continuous multilinear map
    $\mapdef{f}{E_1\times \cdots \times E_n}{F}$,
    for any $(a_1, \ldots, a_n) \in E_1\times \cdots \times E_n$, the
    derivative $\mathrm{D} f_{(a_1, \ldots, a_n)}$ exists and is given by
    \begin{align*}
        \mathrm{D} f_{(a_1, \ldots, a_n)} (u_1, \ldots,  u_n)  & =
        f(u_1, a_2,  a_3, \ldots, a_n) +  f(a_1, u_2, a_3, \ldots, a_n) +
        \cdots \\
        & \quad + f(a_1, a_2, a_3, \ldots, a_{n-1}, u_n)  \\
        & = \sum_{k = 1}^n f(a_1, \ldots, a_{k-1}, u_k, a_{k+1}, \ldots, a_n),
    \end{align*}
    for all $(u_1, \ldots, u_n) \in E_1\times \cdots \times E_n$.\\
    \medskip
    By definition, for every
    $a = (a_1, \ldots, a_n) \in E_1\times \cdots \times E_n$,
    the map $\mathrm{D} f_a$ is a continuous
    linear map from $E_1\times \cdots \times E_n$ to $F$, namely,
    $\mathrm{D} f_a \in \s{L}(E_1\times \cdots \times E_n,  F)$.
    The map
    $\mapdef{\mathrm{D} f}{E_1\times \cdots\times  E_n}
    {\s{L}(E_1\times \cdots \times E_n,  F)}$
    given by $a \mapsto \mathrm{D} f_a$ is
    linear and continuous for $n = 2$,  but it is not linear for $n \geq
    3$.
    It is also not  multilinear for $n \geq 2$, but  it can still be
    shown that  it is  continuous
    (you need not prove it, unless you decide to solve
    the extra credit problem).\\
    \medskip
    Using the above facts, prove (quickly, this is easy) that for {\it any\/} matrix
    $A\in \mathrm{M}_n(\reals)$ and any matrix
    $B\in \mathrm{M}_n(\reals)$, the derivative $d\det_A$ exists and is
    given by
    \begin{align*}
        d\, \mathrm{det}_A (B)  & =
        \det(B^1, A^2,  A^3, \ldots, A^n) +  \det(A^1, B^2, A^3, \ldots, A^n) +
        \cdots \\
        & \quad + \det(A^1, A^2, A^3, \ldots, A^{n-1}, B^n) \\
        & = \sum_{k = 1}^n   \det(A^1, \ldots,A^{k-1}, B^k, A^{k+1},  \ldots, A^{n}),
    \end{align*}
    where $A^1, \ldots, A^n$ are the columns of $A$ and
    $B^1, \ldots, B^n$ are the columns of $B$.
    Furthermore, the map
    $\mapdef{d\det}{\mathrm{M}_n(\reals)}{\s{L}(\mathrm{M}_n(\reals),\reals)}$
    given by $A\mapsto d\det_A$
    is continuous.\\
    \medskip
    Therefore,   $d\det_A$ exists even if $A$ is
    not invertible, but we would like to find a more ``friendly''
    and more explicit expression for it. There
    such an explicit formula involving the adjugate matrix
    $\widetilde{A}$ of $A$ from Section 6.4, Definition 6.9.
}
