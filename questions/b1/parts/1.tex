\question{(B1.1)}{
    Check that $\mathbf{SO}(n, 1)$ is indeed a group with the inverse of $A$  given by
    $A^{-1} = J\transpos{A}J$ (this is the {\it special Lorentz group\/}).
}
\begin{proof}
    We can show that $\mathbf{SO}(n, 1)$ complies with the group axioms
    \begin{itemize}
        \item G1. $\mathbf{SO}(n, 1)$ is associative since matrices are associative
        \item G2. We can see that $I_{n + 1}\in \mathbf{SO}(n, 1)$, since
            \[
                \transpos{I}_{n + 1}JI_{n + 1} = J
            \]
            and this is our identity element.
        \item G3. To prove the provided equation is our inverse, we must first show
            \begin{align*}
                JJ & =
                \begin{pmatrix}
                    I_n & 0 \\
                    0 & -1
                \end{pmatrix}
                \begin{pmatrix}
                    I_n & 0 \\
                    0 & -1
                \end{pmatrix}
                \\ & =
                \begin{pmatrix}
                    I_n & 0 \\
                    0 & (-1)^2
                \end{pmatrix}
                \\ & =
                \begin{pmatrix}
                    I_n & 0 \\
                    0 & 1
                \end{pmatrix}
                \\ & = I_{n + 1}
            \end{align*}
            and since $A\in\mathbf{GL}(n + 1, \reals)$, then $A^{-1}$ must exist. This means we can restate our original question
            like so
            \begin{align*}
                \transpos{A}JA = J = (\transpos{A})^{-1}JA^{-1}.
            \end{align*}
            Now we can show that $A^{-1} = J\transpos{A}J$
            \begin{align*}
                A(J\transpos{A}J) & = A(J\transpos{A}(\transpos{A})^{-1}JA^{-1}) \tag{$J = (\transpos{A})^{-1}JA^{-1}$}
                \\ & = A(JJA^{-1})
                \\ & = AIA^{-1} \tag{$J^2 = I$}
                \\ & = AA^{-1}
                \\ & = I.
            \end{align*}
            So, $A^{-1} = J\transpos{A}J$.
    \end{itemize}
    Since $\mathbf{SO}(n, 1)$ meets all group axioms, we can conclude that it is a group.
\end{proof}
\newpage
